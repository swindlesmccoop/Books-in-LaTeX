\documentclass[oneside]{book}
\title{Art of War}
\author{Sun Tzu\\Translated by Lionel Giles}
\date{May 1994}
\begin{document}
\maketitle

\chapter{Laying Plans}
\begin{enumerate}
    \item Sun Tzu said: The art of war is of vital importance to the State.
    \item It is a matter of life and death, a road either to safety or to ruin. Hence it is a subject of inquiry which can on no account be neglected.
    \item The art of war, then, is governed by five constant factors, to be taken into account in one's deliberations, when seeking to determine the conditions obtaining in the field.
    \item These are: \begin{enumerate}
        \item[1.] The Moral Law
        \item[2.] Heaven
        \item[3.] Earth
        \item[4.] The Commander
        \item[5.] Method and discipline
    \end{enumerate}
    \item[5, 6.] The Moral Law causes the people to be in complete accord with their ruler, so that they will follow him regardless of their lives, undismayed by any danger.
	\setcounter{enumi}{6}
    \item Heaven signifies night and day, cold and heat, times and seasons.
    \item Earth comprises distances, great and small; danger and security; open ground and narrow passes; the chances of life and death.
    \item The Commander stands for the virtues of wisdom, sincerely, benevolence, courage and strictness.
    \item By method and discipline are to be understood the marshaling of the army in its proper subdivisions, the graduations of rank among the officers, the maintenance of roads by which supplies may reach the army, and the control of military expenditure.
    \item These five heads should be familiar to every general: he who knows them will be victorious; he who knows them not will fail.
    \item Therefore, in your deliberations, when seeking to determine the military conditions, let them be made the basis of a comparison, in this wise:
    \item \begin{enumerate}
        \item[1.] Which of the two sovereigns is imbued with the Moral law?
        \item[2.] Which of the two generals has most ability?
        \item[3.] With whom lie the advantages derived from Heaven and Earth?
        \item[4.] On which side is discipline most rigorously enforced?
        \item[5.] Which army is stronger?
        \item[6.] On which side are officers and men more highly trained?
        \item[7.] In which army is there the greater constancy both in reward and punishment?
    \end{enumerate}
    \item By means of these seven considerations I can forecast victory or defeat.
    \item The general that hearkens to my counsel and acts upon it, will conquer: let such a one be retained in command! The general that hearkens not to my counsel nor acts upon it, will suffer defeat:let such a one be dismissed!
    \item While heading the profit of my counsel, avail yourself also of any helpful circumstances over and beyond the ordinary rules.
	\item According as circumstances are favorable, one should modify one's plans.
	\item All warfare is based on deception.
	\item Hence, when able to attack, we must seem unable; when using our forces, we must seem inactive; when we are near, we must make the enemy believe we are far away; when far away, we must make him believe we are near.
	\item Hold out baits to entice the enemy. Feign disorder, and crush him.
	\item If he is secure at all points, be prepared for him. If he is in superior strength, evade him.
	\item If your opponent is of choleric temper, seek to irritate him. Pretend to be weak, that he may grow arrogant.
	\item If he is taking his ease, give him no rest. If his forces are united, separate them.
	\item Attack him where he is unprepared, appear where you are not expected.
	\item These military devices, leading to victory, must not be divulged beforehand.
	\item Now the general who wins a battle makes many calculations in his temple ere the battle is fought. The general who loses a battle makes but few calculations beforehand. Thus do many calculations lead to victory, and few calculations to defeat: how much more no calculation at all! It is by attention to this point that I can foresee who is likely to win or lose.
\end{enumerate}

\chapter{Waging War}
\begin{enumerate}
	\item Sun Tzu said: In the operations of war, where there are in the field a thousand swift chariots, as many heavy chariots, and a hundred thousand mail-clad soldiers, with provisions enough to carry them a thousand li, the expenditure at home and at the front, including entertainment of guests, small items such as glue and paint, and sums spent on chariots and armor, will reach the total of a thousand ounces of silver per day. Such is the cost of raising an army of 100,000 men.
	\item When you engage in actual fighting, if victory is long in coming, then men's weapons will grow dull and their ardor will be damped. If you lay siege to a town, you will exhaust your strength.
	\item Again, if the campaign is protracted, the resources of the State will not be equal to the strain.
	\item Now, when your weapons are dulled, your ardor damped, your strength exhausted and your treasure spent, other chieftains will spring up to take advantage of your extremity. Then no man, however wise, will be able to avert the consequences that must ensue.
	\item Thus, though we have heard of stupid haste in war, cleverness has never been seen associated with long delays.
	\item There is no instance of a country having benefited from prolonged warfare.
	\item It is only one who is thoroughly acquainted with the evils of war that can thoroughly understand the profitable way of carrying it on.
	\item The skillful soldier does not raise a second levy, neither are his supply-wagons loaded more than twice.
	\item Bring war material with you from home, but forage on the enemy. Thus the army will have food enough for its needs.
	\item Poverty of the State exchequer causes an army to be maintained by contributions from a distance. Contributing to maintain an army at a distance causes the people to be impoverished.
	\item On the other hand, the proximity of an army causes prices to go up; and high prices cause the people's substance to be drained away.
	\item When their substance is drained away, the peasantry will be afflicted by heavy exactions.
	\item[13, 14.] With this loss of substance and exhaustion of strength, the homes of the people will be stripped bare, and three-tenths of their income will be dissipated; while government expenses for broken chariots, worn-out horses, breast-plates and helmets, bows and arrows, spears and shields, protective mantles, draught-oxen and heavy wagons, will amount to four-tenths of its total revenue.
	\setcounter{enumi}{14}
	\item Hence a wise general makes a point of foraging on the enemy. One cartload of the enemy's provisions is equivalent to twenty of one's own, and likewise a single picul of his provender is equivalent to twenty from one's own store.
	\item Now in order to kill the enemy, our men must be roused to anger; that there may be advantage from defeating the enemy, they must have their rewards.
	\item Therefore in chariot fighting, when ten or more chariots have been taken, those should be rewarded who took the first. Our own flags should be substituted for those of the enemy, and the chariots mingled and used in conjunction with ours. The captured soldiers should be kindly treated and kept.
	\item This is called, using the conquered foe to augment one's own strength.
	\item In war, then, let your great object be victory, not lengthy campaigns.
	\item Thus it may be known that the leader of armies is the arbiter of the people's fate, the man on whom it depends whether the nation shall be in peace or in peril.
\end{enumerate}

\chapter{Attack by Stratagem}
\begin{enumerate}
	\item Sun Tzu said: In the practical art of war, the best thing of all is to take the enemy's country whole and intact; to shatter and destroy it is not so good. So, too, it is better to recapture an army entire than to destroy it, to capture a regiment, a detachment or a company entire than to destroy them.
	\item Hence to fight and conquer in all your battles is not supreme excellence; supreme excellence consists in breaking the enemy's resistance without fighting.
	\item Thus the highest form of generalship is to balk the enemy's plans; the next best is to prevent the junction of the enemy's forces; the next in order is to attack the enemy's army in the field; and the worst policy of all is to besiege walled cities.
	\item The rule is, not to besiege walled cities if it can possibly be avoided. The preparation of mantlets, movable shelters, and various implements of war, will take up three whole months; and the piling up of mounds over against the walls will take three months more.
	\item The general, unable to control his irritation, will launch his men to the assault like swarming ants, with the result that one-third of his men are slain, while the town still remains untaken. Such are the disastrous effects of a siege.
	\item Therefore the skillful leader subdues the enemy's troops without any fighting; he captures their cities without laying siege to them; he overthrows their kingdom without lengthy operations in the field.
	\item With his forces intact he will dispute the mastery of the Empire, and thus, without losing a man, his triumph will be complete. This is the method of attacking by stratagem.
	\item It is the rule in war, if our forces are ten to the enemy's one, to surround him; if five to one, to attack him; if twice as numerous, to divide our army into two.
	\item If equally matched, we can offer battle; if slightly inferior in numbers, we can avoid the enemy; if quite unequal in every way, we can flee from him.
	\item Hence, though an obstinate fight may be made by a small force, in the end it must be captured by the larger force.
	\item Now the general is the bulwark of the State; if the bulwark is complete at all points; the State will be strong; if the bulwark is defective, the State will be weak.
	\item There are three ways in which a ruler can bring misfortune upon his army:
	\begin{enumerate}
	        \item[1.] By commanding the army to advance or to retreat, being ignorant of the fact that it cannot obey. This is called hobbling the army.
	        \item[2.] By attempting to govern an army in the same way as he administers a kingdom, being ignorant of the conditions which obtain in an army. This causes restlessness in the soldier's minds.
	        \item[3.] By employing the officers of his army without discrimination, through ignorance of the military principle of adaptation to circumstances. This shakes the confidence of the soldiers.
	        \item[4.] But when the army is restless and distrustful, trouble is sure to come from the other feudal princes. This is simply bringing anarchy into the army, and flinging victory away.
	\end{enumerate}
	\item Thus we may know that there are five essentials for victory:
	\begin{enumerate}
		\item[1.] He will win who knows when to fight and when not to fight.
		\item[2.] He will win who knows how to handle both superior and inferior forces.
		\item[3.] He will win whose army is animated by the same spirit throughout all its ranks.
		\item[4.] He will win who, prepared himself, waits to take the enemy unprepared.
		\item[5.] He will win who has military capacity and is not interfered with by the sovereign.
	\end{enumerate}
	\item Hence the saying: If you know the enemy and know yourself, you need not fear the result of a hundred battles. If you know yourself but not the enemy, for every victory gained you will also suffer a defeat. If you know neither the enemy nor yourself, you will succumb in every battle.
\end{enumerate}

\chapter{Tactical Dispositions}
\begin{enumerate}
    \item Sun Tzu said: The good fighters of old first put themselves beyond the possibility of defeat, and then waited for an opportunity of defeating the enemy.
    \item To secure ourselves against defeat lies in our own hands, but the opportunity of defeating the enemy is provided by the enemy himself.
    \item Thus the good fighter is able to secure himself against defeat, but cannot make certain of defeating the enemy.
	\item Security against defeat implies defensive tactics; ability to defeat the enemy means taking the offensive.
	\item Standing on the defensive indicates insufficient strength; attacking, a superabundance of strength.
	\item The general who is skilled in defense hides in the most secret recesses of the earth; he who is skilled in attack flashes forth from the topmost heights of heaven. Thus on the one hand we have ability to protect ourselves; on the other, a victory that is complete.
	\item To see victory only when it is within the ken of the common herd is not the acme of excellence.
	\item Neither is it the acme of excellence if you fight and conquer and the whole Empire says, "Well done!"
	\item To lift an autumn hair is no sign of great strength; to see the sun and moon is no sign of sharp sight; to hear the noise of thunder is no sign of a quick ear.
	\item What the ancients called a clever fighter is one who not only wins, but excels in winning with ease.
	\item Hence his victories bring him neither reputation for wisdom nor credit for courage.
	\item He wins his battles by making no mistakes. Making no mistakes is what establishes the certainty of victory, for it means conquering an enemy that is already defeated.
	\item Hence the skillful fighter puts himself into a position which makes defeat impossible, and does not miss the moment for defeating the enemy.
	\item Thus it is that in war the victorious strategist only seeks battle after the victory has been won, whereas he who is destined to defeat first fights and afterwards looks for victory.
	\item The consummate leader cultivates the moral law, and strictly adheres to method and discipline; thus it is in his power to control success.
	\item In respect of military method, we have, firstly, Measurement; secondly, Estimation of quantity; thirdly, Calculation; fourthly, Balancing of chances; fifthly, Victory.
	\item Measurement owes its existence to Earth; Estimation of quantity to Measurement; Calculation to Estimation of quantity; Balancing of chances to Calculation; and Victory to Balancing of chances.
	\item A victorious army opposed to a routed one, is as a pound's weight placed in the scale against a single grain.
	\item The onrush of a conquering force is like the bursting of pent-up waters into a chasm a thousand fathoms deep.
\end{enumerate}

\chapter{Energy}
\begin{enumerate}
	\item Sun Tzu said: The control of a large force is the same principle as the control of a few men: it is merely a question of dividing up their numbers.
	\item Fighting with a large army under your command is nowise different from fighting with a small one: it is merely a question of instituting signs and signals.
	\item To ensure that your whole host may withstand the brunt of the enemy's attack and remain unshaken – this is effected by maneuvers direct and indirect.
	\item That the impact of your army may be like a grindstone dashed against an egg – this is effected by the science of weak points and strong.
	\item In all fighting, the direct method may be used for joining battle, but indirect methods will be needed in order to secure victory.
	\item Indirect tactics, efficiently applied, are inexhaustible as Heaven and Earth, unending as the flow of rivers and streams; like the sun and moon, they end but to begin anew; like the four seasons, they pass away to return once more.
	\item There are not more than five musical notes, yet the combinations of these five give rise to more melodies than can ever be heard.
	\item There are not more than five primary colors (blue, yellow, red, white, and black), yet in combination they produce more hues than can ever been seen.
	\item There are not more than five cardinal tastes (sour, acrid, salt, sweet, bitter), yet combinations of them yield more flavors than can ever be tasted.
	\item In battle, there are not more than two methods of attack – the direct and the indirect; yet these two in combination give rise to an endless series of maneuvers.
	\item The direct and the indirect lead on to each other in turn. It is like moving in a circle – you never come to an end. Who can exhaust the possibilities of their combination?
	\item The onset of troops is like the rush of a torrent which will even roll stones along in its course.
	\item The quality of decision is like the well-timed swoop of a falcon which enables it to strike and destroy its victim.
	\item Therefore the good fighter will be terrible in his onset, and prompt in his decision.
	\item Energy may be likened to the bending of a crossbow; decision, to the releasing of a trigger.
	\item Amid the turmoil and tumult of battle, there may be seeming disorder and yet no real disorder at all; amid confusion and chaos, your array may be without head or tail, yet it will be proof against defeat.
	\item Simulated disorder postulates perfect discipline, simulated fear postulates courage; simulated weakness postulates strength.
	\item Hiding order beneath the cloak of disorder is simply a question of subdivision; concealing courage under a show of timidity presupposes a fund of latent energy; masking strength with weakness is to be effected by tactical dispositions.
	\item Thus one who is skillful at keeping the enemy on the move maintains deceitful appearances, according to which the enemy will act. He sacrifices something, that the enemy may snatch at it.
	\item By holding out baits, he keeps him on the march; then with a body of picked men he lies in wait for him.
	\item The clever combatant looks to the effect of combined energy, and does not require too much from individuals. Hence his ability to pick out the right men and utilize combined energy.
	\item When he utilizes combined energy, his fighting men become as it were like unto rolling logs or stones. For it is the nature of a log or stone to remain motionless on level ground, and to move when on a slope; if four-cornered, to come to a standstill, but if round-shaped, to go rolling down.
	\item Thus the energy developed by good fighting men is as the momentum of a round stone rolled down a mountain thousands of feet in height. So much on the subject of energy.
\end{enumerate}

\chapter{Weak Points and Strong}
\begin{enumerate}
	\item Sun Tzu said: Whoever is first in the field and awaits the coming of the enemy, will be fresh for the fight; whoever is second in the field and has to hasten to battle will arrive exhausted.
	\item Therefore the clever combatant imposes his will on the enemy, but does not allow the enemy's will to be imposed on him.
	\item By holding out advantages to him, he can cause the enemy to approach of his own accord; or, by inflicting damage, he can make it impossible for the enemy to draw near.
	\item If the enemy is taking his ease, he can harass him; if well supplied with food, he can starve him out; if quietly encamped, he can force him to move.
	\item Appear at points which the enemy must hasten to defend; march swiftly to places where you are not expected.
	\item An army may march great distances without distress, if it marches through country where the enemy is not.
	\item You can be sure of succeeding in your attacks if you only attack places which are undefended.You can ensure the safety of your defense if you only hold positions that cannot be attacked.
	\item Hence that general is skillful in attack whose opponent does not know what to defend; and he is skillful in defense whose opponent does not know what to attack.
	\item O divine art of subtlety and secrecy! Through you we learn to be invisible, through you inaudible; and hence we can hold the enemy's fate in our hands.
	\item You may advance and be absolutely irresistible, if you make for the enemy's weak points; you may retire and be safe from pursuit if your movements are more rapid than those of the enemy.
	\item If we wish to fight, the enemy can be forced to an engagement even though he be sheltered behind a high rampart and a deep ditch. All we need do is attack some other place that he will be obliged to relieve.
	\item If we do not wish to fight, we can prevent the enemy from engaging us even though the lines of our encampment be merely traced out on the ground. All we need do is to throw something odd and unaccountable in his way.
	\item By discovering the enemy's dispositions and remaining invisible ourselves, we can keep our forces concentrated, while the enemy's must be divided.
	\item We can form a single united body, while the enemy must split up into fractions. Hence there will be a whole pitted against separate parts of a whole, which means that we shall be many to the enemy's few.
	\item And if we are able thus to attack an inferior force with a superior one, our opponents will be in dire straits.
	\item The spot where we intend to fight must not be made known; for then the enemy will have to prepare against a possible attack at several different points; and his forces being thus distributed in many directions, the numbers we shall have to face at any given point will be proportionately few.
	\item For should the enemy strengthen his van, he will weaken his rear; should he strengthen his rear, he will weaken his van; should he strengthen his left, he will weaken his right; should he strengthen his right, he will weaken his left. If he sends reinforcements everywhere, he will everywhere be weak.
	\item Numerical weakness comes from having to prepare against possible attacks; numerical strength, from compelling our adversary to make these preparations against us.
	\item Knowing the place and the time of the coming battle, we may concentrate from the greatest distances in order to fight.
	\item But if neither time nor place be known, then the left wing will be impotent to succor the right, the right equally impotent to succor the left, the van unable to relieve the rear, or the rear to support the van. How much more so if the furthest portions of the army are anything under a hundred LI apart, and even the nearest are separated by several LI!
	\item Though according to my estimate the soldiers of Yueh exceed our own in number, that shall advantage them nothing in the matter of victory. I say then that victory can be achieved.
	\item Though the enemy be stronger in numbers, we may prevent him from fighting. Scheme so as to discover his plans and the likelihood of their success.
	\item Rouse him, and learn the principle of his activity or inactivity. Force him to reveal himself, so as to find out his vulnerable spots.
	\item Carefully compare the opposing army with your own, so that you may know where strength is superabundant and where it is deficient.
	\item In making tactical dispositions, the highest pitch you can attain is to conceal them; conceal your dispositions, and you will be safe from the prying of the subtlest spies, from the machinations of the wisest brains.
	\item How victory may be produced for them out of the enemy's own tactics – that is what the multitude cannot comprehend.
	\item All men can see the tactics whereby I conquer, but what none can see is the strategy out of which victory is evolved.
	\item Do not repeat the tactics which have gained you one victory, but let your methods be regulated by the infinite variety of circumstances.
	\item Military tactics are like unto water; for water in its natural course runs away from high places and hastens downwards.
	\item So in war, the way is to avoid what is strong and to strike at what is weak.
	\item Water shapes its course according to the nature of the ground over which it flows; the soldier works out his victory in relation to the foe whom he is facing.
	\item Therefore, just as water retains no constant shape, so in warfare there are no constant conditions.
	\item He who can modify his tactics in relation to his opponent and thereby succeed in winning, may be called a heaven-born captain.
	\item The five elements (water, fire, wood, metal, earth) are not always equally predominant; the four seasons make way for each other in turn. There are short days and long; the moon has its periods of waning and waxing.
\end{enumerate}

\chapter{Maneuvering}
\begin{enumerate}
	\item Sun Tzu said: In war, the general receives his commands from the sovereign.
	\item Having collected an army and concentrated his forces, he must blend and harmonize the different elements thereof before pitching his camp.
	\item After that, comes tactical maneuvering, than which there is nothing more difficult. The difficulty of tactical maneuvering consists in turning the devious into the direct, and misfortune into gain.
	\item Thus, to take a long and circuitous route, after enticing the enemy out of the way, and though starting after him, to contrive to reach the goal before him, shows knowledge of the artifice of deviation.
	\item Maneuvering with an army is advantageous; with an undisciplined multitude, most dangerous.
	\item If you set a fully equipped army in march in order to snatch an advantage, the chances are that you will be too late. On the other hand, to detach a flying column for the purpose involves the sacrifice of its baggage and stores.
	\item Thus, if you order your men to roll up their buff-coats, and make forced marches without halting day or night, covering double the usual distance at a stretch, doing a hundred LI in order to wrest an advantage, the leaders of all your three divisions will fall into the hands of the enemy.
	\item The stronger men will be in front, the jaded ones will fall behind, and on this plan only one-tenth of your army will reach its destination.
	\item If you march fifty LI in order to outmaneuver the enemy, you will lose the leader of your first division, and only half your force will reach the goal.
	\item If you march thirty LI with the same object, two-thirds of your army will arrive.
	\item We may take it then that an army without its baggage-train is lost; without provisions it is lost; without bases of supply it is lost.
	\item We cannot enter into alliances until we are acquainted with the designs of our neighbors.
	\item We are not fit to lead an army on the march unless we are familiar with the face of the country – its mountains and forests, its pitfalls and precipices, its marshes and swamps.
	\item We shall be unable to turn natural advantage to account unless we make use of local guides.
	\item In war, practice dissimulation, and you will succeed.
	\item Whether to concentrate or to divide your troops, must be decided by circumstances.
	\item Let your rapidity be that of the wind, your compactness that of the forest.
	\item In raiding and plundering be like fire, is immovability like a mountain.
	\item Let your plans be dark and impenetrable as night, and when you move, fall like a thunderbolt.
	\item When you plunder a countryside, let the spoil be divided amongst your men; when you capture new territory, cut it up into allotments for the benefit of the soldiery.
	\item Ponder and deliberate before you make a move.
	\item He will conquer who has learnt the artifice of deviation. Such is the art of maneuvering.
	\item The Book of Army Management says: On the field of battle, the spoken word does not carry far enough: hence the institution of gongs and drums. Nor can ordinary objects be seen clearly enough: hence the institution of banners and flags.
	\item Gongs and drums, banners and flags, are means whereby the ears and eyes of the host may be focused on one particular point.
	\item The host thus forming a single united body, is it impossible either for the brave to advance alone, or for the cowardly to retreat alone. This is the art of handling large masses of men.
	\item In night-fighting, then, make much use of signal-fires and drums, and in fighting by day, of flags and banners, as a means of influencing the ears and eyes of your army.
	\item A whole army may be robbed of its spirit; a commander-in-chief may be robbed of his presence of mind.
	\item Now a soldier's spirit is keenest in the morning; by noonday it has begun to flag; and in the evening, his mind is bent only on returning to camp.
	\item A clever general, therefore, avoids an army when its spirit is keen, but attacks it when it is sluggish and inclined to return. This is the art of studying moods.
	\item Disciplined and calm, to await the appearance of disorder and hubbub amongst the enemy – this is the art of retaining self-possession.
	\item To be near the goal while the enemy is still far from it, to wait at ease while the enemy is toiling and struggling, to be well-fed while the enemy is famished – this is the art of husbanding one's strength.
	\item To refrain from intercepting an enemy whose banners are in perfect order, to refrain from attacking an army drawn up in calm and confident array – this is the art of studying circumstances.
	\item It is a military axiom not to advance uphill against the enemy, nor to oppose him when he comes downhill.
	\item Do not pursue an enemy who simulates flight; do not attack soldiers whose temper is keen.
	\item Do not swallow bait offered by the enemy. Do not interfere with an army that is returning home.
	\item When you surround an army, leave an outlet free. Do not press a desperate foe too hard.
	\item Such is the art of warfare.
\end{enumerate}

\chapter{Variation in Tactics}
\begin{enumerate}
	\item Sun Tzu said: In war, the general receives his commands from the sovereign, collects his army and concentrates his forces
	\item When in difficult country, do not encamp. In country where high roads intersect, join hands with your allies. Do not linger in dangerously isolated positions. In hemmed-in situations, you must resort to stratagem. In desperate position, you must fight.
	\item There are roads which must not be followed, armies which must be not attacked, towns which must be besieged, positions which must not be contested, commands of the sovereign which must not be obeyed.
	\item The general who thoroughly understands the advantages that accompany variation of tactics knows how to handle his troops.
	\item The general who does not understand these, may be well acquainted with the configuration of the country, yet he will not be able to turn his knowledge to practical account.
	\item So, the student of war who is unversed in the art of war of varying his plans, even though he be acquainted with the Five Advantages, will fail to make the best use of his men.
	\item Hence in the wise leader's plans, considerations of advantage and of disadvantage will be blended together.
	\item If our expectation of advantage be tempered in this way, we may succeed in accomplishing the essential part of our schemes.
	\item If, on the other hand, in the midst of difficulties we are always ready to seize an advantage, we may extricate ourselves from misfortune.
	\item Reduce the hostile chiefs by inflicting damage on them; and make trouble for them, and keep them constantly engaged; hold out specious allurements, and make them rush to any given point.
	\item The art of war teaches us to rely not on the likelihood of the enemy's not coming, but on our own readiness to receive him; not on the chance of his not attacking, but rather on the fact that we have made our position unassailable.
	\item There are five dangerous faults which may affect a general:
	\begin{enumerate}
	    \item[1.] Recklessness, which leads to destruction.
	    \item[2.] Cowardice, which leads to capture.
	    \item[3.] A hasty temper, which can be provoked by insults.
	    \item[4.] A delicacy of honor which is sensitive to shame.
	    \item[5.] Over-solicitude for his men, which exposes him to worry and trouble.
	\end{enumerate}
	\item These are the five besetting sins of a general, ruinous to the conduct of war.
	\item When an army is overthrown and its leader slain, the cause will surely be found among these five dangerous faults. Let them be a subject of meditation.
\end{enumerate}

\chapter{The Army on the March}
\begin{enumerate}
	\item Sun Tzu said: We come now to the question of encamping the army, and observing signs of the enemy. Pass quickly over mountains, and keep in the neighborhood of valleys.
	\item Camp in high places, facing the sun. Do not climb heights in order to fight. So much for mountain warfare.
	\item After crossing a river, you should get far away from it.
	\item When an invading force crosses a river in its onward march, do not advance to meet it in mid-stream. It will be best to let half the army get across, and then deliver your attack.
	\item If you are anxious to fight, you should not go to meet the invader near a river which he has to cross.
	\item Moor your craft higher up than the enemy, and facing the sun. Do not move up-stream to meet the enemy. So much for river warfare.
	\item In crossing salt-marshes, your sole concern should be to get over them quickly, without any delay.
	\item If forced to fight in a salt-marsh, you should have water and grass near you, and get your back to a clump of trees. So much for operations in salt-marches.
	\item In dry, level country, take up an easily accessible position with rising ground to your right and on your rear, so that the danger may be in front, and safety lie behind. So much for campaigning in flat country.
	\item These are the four useful branches of military knowledge which enabled the Yellow Emperor to vanquish four several sovereigns.
	\item All armies prefer high ground to low and sunny places to dark.
	\item If you are careful of your men, and camp on hard ground, the army will be free from disease of every kind, and this will spell victory.
	\item When you come to a hill or a bank, occupy the sunny side, with the slope on your right rear. Thus you will at once act for the benefit of your soldiers and utilize the natural advantages of the ground.
	\item When, in consequence of heavy rains up-country, a river which you wish to ford is swollen and flecked with foam, you must wait until it subsides.
	\item Country in which there are precipitous cliffs with torrents running between, deep natural hollows, confined places, tangled thickets, quagmires and crevasses, should be left with all possible speed and not approached.
	\item While we keep away from such places, we should get the enemy to approach them; while we face them, we should let the enemy have them on his rear.
	\item If in the neighborhood of your camp there should be any hilly country, ponds surrounded by aquatic grass, hollow basins filled with reeds, or woods with thick undergrowth, they must be carefully routed out and searched; for these are places where men in ambush or insidious spies are likely to be lurking.
	\item When the enemy is close at hand and remains quiet, he is relying on the natural strength of his position.
	\item When he keeps aloof and tries to provoke a battle, he is anxious for the other side to advance.
	\item If his place of encampment is easy of access, he is tendering a bait.
	\item Movement amongst the trees of a forest shows that the enemy is advancing. The appearance of a number of screens in the midst of thick grass means that the enemy wants to make us suspicious.
	\item The rising of birds in their flight is the sign of an ambuscade. Startled beasts indicate that a sudden attack is coming.
	\item When there is dust rising in a high column, it is the sign of chariots advancing; when the dust is low, but spread over a wide area, it betokens the approach of infantry. When it branches out in different directions, it shows that parties have been sent to collect firewood. A few clouds of dust moving to and fro signify that the army is encamping.
	\item Humble words and increased preparations are signs that the enemy is about to advance. Violent language and driving forward as if to the attack are signs that he will retreat.
	\item When the light chariots come out first and take up a position on the wings, it is a sign that the enemy is forming for battle.
	\item Peace proposals unaccompanied by a sworn covenant indicate a plot.
	\item When there is much running about and the soldiers fall into rank, it means that the critical moment has come.
	\item When some are seen advancing and some retreating, it is a lure.
	\item When the soldiers stand leaning on their spears, they are faint from want of food.
	\item If those who are sent to draw water begin by drinking themselves, the army is suffering from thirst.
	\item If the enemy sees an advantage to be gained and makes no effort to secure it, the soldiers are exhausted.
	\item If birds gather on any spot, it is unoccupied. Clamor by night betokens nervousness.
	\item If there is disturbance in the camp, the general's authority is weak. If the banners and flags are shifted about, sedition is afoot. If the officers are angry, it means that the men are weary.
	\item When an army feeds its horses with grain and kills its cattle for food, and when the men do not hang their cooking-pots over the camp-fires, showing that they will not return to their tents, you may know that they are determined to fight to the death.
	\item The sight of men whispering together in small knots or speaking in subdued tones points to disaffection amongst the rank and file.
	\item Too frequent rewards signify that the enemy is at the end of his resources; too many punishments betray a condition of dire distress.
	\item To begin by bluster, but afterwards to take fright at the enemy's numbers, shows a supreme lack of intelligence.
	\item When envoys are sent with compliments in their mouths, it is a sign that the enemy wishes for a truce.
	\item If the enemy's troops march up angrily and remain facing ours for a long time without either joining battle or taking themselves off again, the situation is one that demands great vigilance and circumspection.
	\item If our troops are no more in number than the enemy, that is amply sufficient; it only means that no direct attack can be made. What we can do is simply to concentrate all our available strength, keep a close watch on the enemy, and obtain reinforcements.
	\item He who exercises no forethought but makes light of his opponents is sure to be captured by them.
	\item If soldiers are punished before they have grown attached to you, they will not prove submissive; and, unless submissive, then will be practically useless. If, when the soldiers have become attached to you, punishments are not enforced, they will still be unless.
	\item Therefore soldiers must be treated in the first instance with humanity, but kept under control by means of iron discipline. This is a certain road to victory.
	\item If in training soldiers commands are habitually enforced, the army will be well-disciplined; if not, its discipline will be bad.
	\item If a general shows confidence in his men but always insists on his orders being obeyed, the gain will be mutual.
\end{enumerate}

\chapter{Terrain}
\begin{enumerate}
	\item Sun Tzu said: We may distinguish six kinds of terrain, to wit:
	\begin{enumerate}
		\item[1.] Accessible ground
		\item[2.] Entangling ground
		\item[3.] Temporizing ground
		\item[4.] Narrow passes
		\item[5.] Precipitous heights
	\end{enumerate}
	\item Positions at a great distance from the enemy.
	\item Ground which can be freely traversed by both sides is called accessible.
	\item With regard to ground of this nature, be before the enemy in occupying the raised and sunny spots, and carefully guard your line of supplies. Then you will be able to fight with advantage.
	\item Ground which can be abandoned but is hard to re-occupy is called entangling.
	\item From a position of this sort, if the enemy is unprepared, you may sally forth and defeat him. But if the enemy is prepared for your coming, and you fail to defeat him, then, return being impossible, disaster will ensue.
	\item When the position is such that neither side will gain by making the first move, it is called temporizing ground.
	\item In a position of this sort, even though the enemy should offer us an attractive bait, it will be advisable not to stir forth, but rather to retreat, thus enticing the enemy in his turn; then, when part of his army has come out, we may deliver our attack with advantage.
	\item With regard to narrow passes, if you can occupy them first, let them be strongly garrisoned and await the advent of the enemy.
	\item Should the army forestall you in occupying a pass, do not go after him if the pass is fully garrisoned, but only if it is weakly garrisoned.
	\item With regard to precipitous heights, if you are beforehand with your adversary, you should occupy the raised and sunny spots, and there wait for him to come up.
	\item If the enemy has occupied them before you, do not follow him, but retreat and try to entice him away.
	\item If you are situated at a great distance from the enemy, and the strength of the two armies is equal, it is not easy to provoke a battle, and fighting will be to your disadvantage.
	\item These six are the principles connected with Earth. The general who has attained a responsible post must be careful to study them.
	\item Now an army is exposed to six several calamities, not arising from natural causes, but from faults for which the general is responsible. These are:
	\begin{enumerate}
	    \item[1.] Flight
	    \item[2.] Insubordination
	    \item[3.] Collapse
	    \item[4.] Ruin
	    \item[5.] Disorganization
	    \item[6.] Rout
	\end{enumerate}
	\item Other conditions being equal, if one force is hurled against another ten times its size, the result will be the flight of the former.
	\item When the common soldiers are too strong and their officers too weak, the result is insubordination. When the officers are too strong and the common soldiers too weak, the result is collapse.
	\item When the higher officers are angry and insubordinate, and on meeting the enemy give battle on their own account from a feeling of resentment, before the commander-in-chief can tell whether or no he is in a position to fight, the result is ruin.
	\item When the general is weak and without authority; when his orders are not clear and distinct; when there are no fixes duties assigned to officers and men, and the ranks are formed in a slovenly haphazard manner, the result is utter disorganization.
	\item When a general, unable to estimate the enemy's strength, allows an inferior force to engage a larger one, or hurls a weak detachment against a powerful one, and neglects to place picked soldiers in the front rank, the result must be rout.
	\item These are six ways of courting defeat, which must be carefully noted by the general who has attained a responsible post.
	\item The natural formation of the country is the soldier's best ally; but a power of estimating the adversary, of controlling the forces of victory, and of shrewdly calculating difficulties, dangers and distances, constitutes the test of a great general.
	\item He who knows these things, and in fighting puts his knowledge into practice, will win his battles. He who knows them not, nor practices them, will surely be defeated.
	\item If fighting is sure to result in victory, then you must fight, even though the ruler forbid it; if fighting will not result in victory, then you must not fight even at the ruler's bidding.
	\item The general who advances without coveting fame and retreats without fearing disgrace, whose only thought is to protect his country and do good service for his sovereign, is the jewel of the kingdom.
	\item Regard your soldiers as your children, and they will follow you into the deepest valleys; look upon them as your own beloved sons, and they will stand by you even unto death.
	\item If, however, you are indulgent, but unable to make your authority felt; kind-hearted, but unable to enforce your commands; and incapable, moreover, of quelling disorder: then your soldiers must be likened to spoilt children; they are useless for any practical purpose.
	\item If we know that our own men are in a condition to attack, but are unaware that the enemy is not open to attack, we have gone only halfway towards victory.
	\item If we know that the enemy is open to attack, but are unaware that our own men are not in a condition to attack, we have gone only halfway towards victory.
	\item If we know that the enemy is open to attack, and also know that our men are in a condition to attack, but are unaware that the nature of the ground makes fighting impracticable, we have still gone only halfway towards victory.
	\item Hence the experienced soldier, once in motion, is never bewildered; once he has broken camp, he is never at a loss.
	\item Hence the saying: If you know the enemy and know yourself, your victory will not stand in doubt; if you know Heaven and know Earth, you may make your victory complete.
\end{enumerate}

\chapter{The Nine Situations}
\begin{enumerate}
	\item Sun Tzu said: The art of war recognizes nine varieties of ground:
	\begin{enumerate}
	    \item[1.] Dispersive ground
	    \item[2.] Facile ground
	    \item[3.] Contentious ground
	    \item[4.] Open ground
	    \item[5.] Ground of intersecting highways
	    \item[6.] Serious ground
	    \item[7.] Difficult ground
	    \item[8.] Hemmed-in ground
	    \item[9.] Desperate ground
	\end{enumerate}
	\item When a chieftain is fighting in his own territory, it is dispersive ground.
	\item When he has penetrated into hostile territory, but to no great distance, it is facile ground.
	\item Ground the possession of which imports great advantage to either side, is contentious ground.
	\item Ground on which each side has liberty of movement is open ground.
	\item Ground which forms the key to three contiguous states, so that he who occupies it first has most of the Empire at his command, is a ground of intersecting highways.
	\item When an army has penetrated into the heart of a hostile country, leaving a number of fortified cities in its rear, it is serious ground.
	\item Mountain forests, rugged steeps, marshes and fens – all country that is hard to traverse: this is difficult ground.
	\item Ground which is reached through narrow gorges, and from which we can only retire by tortuous paths, so that a small number of the enemy would suffice to crush a large body of our men: this is hemmed in ground.
	\item Ground on which we can only be saved from destruction by fighting without delay, is desperate ground.
	\item On dispersive ground, therefore, fight not. On facile ground, halt not. On contentious ground, attack not.
	\item On open ground, do not try to block the enemy's way. On the ground of intersecting highways, join hands with your allies.
	\item On serious ground, gather in plunder. In difficult ground, keep steadily on the march.
	\item On hemmed-in ground, resort to stratagem. On desperate ground, fight.
	\item Those who were called skillful leaders of old knew how to drive a wedge between the enemy's front and rear; to prevent co-operation between his large and small divisions; to hinder the good troops from rescuing the bad, the officers from rallying their men.
	\item When the enemy's men were united, they managed to keep them in disorder.
	\item When it was to their advantage, they made a forward move; when otherwise, they stopped still.
	\item If asked how to cope with a great host of the enemy in orderly array and on the point of marching to the attack, I should say: "Begin by seizing something which your opponent holds dear; then he will be amenable to your will."
	\item Rapidity is the essence of war: take advantage of the enemy's unreadiness, make your way by unexpected routes, and attack unguarded spots.
	\item The following are the principles to be observed by an invading force: The further you penetrate into a country, the greater will be the solidarity of your troops, and thus the defenders will not prevail against you.
	\item Make forays in fertile country in order to supply your army with food.
	\item Carefully study the well-being of your men, and do not overtax them. Concentrate your energy and hoard your strength. Keep your army continually on the move, and devise unfathomable plans.
	\item Throw your soldiers into positions whence there is no escape, and they will prefer death to flight. If they will face death, there is nothing they may not achieve. Officers and men alike will put forth their uttermost strength.
	\item Soldiers when in desperate straits lose the sense of fear. If there is no place of refuge, they will stand firm. If they are in hostile country, they will show a stubborn front. If there is no help for it, they will fight hard.
	\item Thus, without waiting to be marshaled, the soldiers will be constantly on the qui vive; without waiting to be asked, they will do your will; without restrictions, they will be faithful; without giving orders, they can be trusted.
	\item Prohibit the taking of omens, and do away with superstitious doubts. Then, until death itself comes, no calamity need be feared.
	\item If our soldiers are not overburdened with money, it is not because they have a distaste for riches; if their lives are not unduly long, it is not because they are disinclined to longevity.
	\item On the day they are ordered out to battle, your soldiers may weep, those sitting up bedewing their garments, and those lying down letting the tears run down their cheeks. But let them once be brought to bay, and they will display the courage of a Chu or a Kuei.
	\item The skillful tactician may be likened to the shuai-jan. Now the shuai-jan is a snake that is found in the ChUng mountains. Strike at its head, and you will be attacked by its tail; strike at its tail, and you will be attacked by its head; strike at its middle, and you will be attacked by head and tail both.
	\item Asked if an army can be made to imitate the shuai-jan, I should answer, Yes. For the men of Wu and the men of Yueh are enemies; yet if they are crossing a river in the same boat and are caught by a storm, they will come to each other's assistance just as the left hand helps the right.
	\item Hence it is not enough to put one's trust in the tethering of horses, and the burying of chariot wheels in the ground
	\item The principle on which to manage an army is to set up one standard of courage which all must reach.
	\item How to make the best of both strong and weak – that is a question involving the proper use of ground.
	\item Thus the skillful general conducts his army just as though he were leading a single man, willy-nilly, by the hand.
	\item It is the business of a general to be quiet and thus ensure secrecy; upright and just, and thus maintain order.
	\item He must be able to mystify his officers and men by false reports and appearances, and thus keep them in total ignorance.
	\item By altering his arrangements and changing his plans, he keeps the enemy without definite knowledge. By shifting his camp and taking circuitous routes, he prevents the enemy from anticipating his purpose.
	\item At the critical moment, the leader of an army acts like one who has climbed up a height and then kicks away the ladder behind him. He carries his men deep into hostile territory before he shows his hand.
	\item He burns his boats and breaks his cooking-pots; like a shepherd driving a flock of sheep, he drives his men this way and that, and nothing knows whither he is going.
	\item To muster his host and bring it into danger – this may be termed the business of the general.
	\item The different measures suited to the nine varieties of ground; the expediency of aggressive or defensive tactics; and the fundamental laws of human nature: these are things that must most certainly be studied.
	\item When invading hostile territory, the general principle is, that penetrating deeply brings cohesion; penetrating but a short way means dispersion.
	\item When you leave your own country behind, and take your army across neighborhood territory, you find yourself on critical ground. When there are means of communication on all four sides, the ground is one of intersecting highways.
	\item When you penetrate deeply into a country, it is serious ground. When you penetrate but a little way, it is facile ground.
	\item When you have the enemy's strongholds on your rear, and narrow passes in front, it is hemmed-in ground. When there is no place of refuge at all, it is desperate ground.
	\item Therefore, on dispersive ground, I would inspire my men with unity of purpose. On facile ground, I would see that there is close connection between all parts of my army.
	\item On contentious ground, I would hurry up my rear.
	\item On open ground, I would keep a vigilant eye on my defenses. On ground of intersecting highways, I would consolidate my alliances.
	\item On serious ground, I would try to ensure a continuous stream of supplies. On difficult ground, I would keep pushing on along the road.
	\item On hemmed-in ground, I would block any way of retreat. On desperate ground, I would proclaim to my soldiers the hopelessness of saving their lives.
	\item For it is the soldier's disposition to offer an obstinate resistance when surrounded, to fight hard when he cannot help himself, and to obey promptly when he has fallen into danger.
	\item We cannot enter into alliance with neighboring princes until we are acquainted with their designs. We are not fit to lead an army on the march unless we are familiar with the face of the country – its mountains and forests, its pitfalls and precipices, its marshes and swamps. We shall be unable to turn natural advantages to account unless we make use of local guides.
	\item To be ignored of any one of the following four or five principles does not befit a warlike prince.
	\item When a warlike prince attacks a powerful state, his generalship shows itself in preventing the concentration of the enemy's forces. He overawes his opponents, and their allies are prevented from joining against him.
	\item Hence he does not strive to ally himself with all and sundry, nor does he foster the power of other states. He carries out his own secret designs, keeping his antagonists in awe. Thus he is able to capture their cities and overthrow their kingdoms.
	\item Bestow rewards without regard to rule, issue orders without regard to previous arrangements; and you will be able to handle a whole army as though you had to do with but a single man.
	\item Confront your soldiers with the deed itself; never let them know your design. When the outlook is bright, bring it before their eyes; but tell them nothing when the situation is gloomy.
	\item Place your army in deadly peril, and it will survive; plunge it into desperate straits, and it will come off in safety.
	\item For it is precisely when a force has fallen into harm's way that is capable of striking a blow for victory.
	\item Success in warfare is gained by carefully accommodating ourselves to the enemy's purpose.
	\item By persistently hanging on the enemy's flank, we shall succeed in the long run in killing the commander-in-chief.
	\item This is called ability to accomplish a thing by sheer cunning.
	\item On the day that you take up your command, block the frontier passes, destroy the official tallies, and stop the passage of all emissaries.
	\item Be stern in the council-chamber, so that you may control the situation.
	\item If the enemy leaves a door open, you must rush in.
	\item Forestall your opponent by seizing what he holds dear, and subtly contrive to time his arrival on the ground.
	\item Walk in the path defined by rule, and accommodate yourself to the enemy until you can fight a decisive battle.
	\item At first, then, exhibit the coyness of a maiden, until the enemy gives you an opening; afterwards emulate the rapidity of a running hare, and it will be too late for the enemy to oppose you.
\end{enumerate}

\chapter{The Attack by Fire}
\begin{enumerate}
	\item Sun Tzu said: There are five ways of attacking with fire. The first is to burn soldiers in their camp; the second is to burn stores; the third is to burn baggage trains; the fourth is to burn arsenals and magazines; the fifth is to hurl dropping fire amongst the enemy.
	\item In order to carry out an attack, we must have means available. The material for raising fire should always be kept in readiness.
	\item There is a proper season for making attacks with fire, and special days for starting a conflagration.
	\item The proper season is when the weather is very dry; the special days are those when the moon is in the constellations of the Sieve, the Wall, the Wing or the Cross-bar; for these four are all days of rising wind.
	\item In attacking with fire, one should be prepared to meet five possible developments:
	\begin{enumerate}
		\item[1] When fire breaks out inside to enemy's camp, respond at once with an attack from without.
		\item[2] If there is an outbreak of fire, but the enemy's soldiers remain quiet, bide your time and do not attack.
		\item[3] When the force of the flames has reached its height, follow it up with an attack, if that is practicable; if not, stay where you are.
		\item[4] If it is possible to make an assault with fire from without, do not wait for it to break out within, but deliver your attack at a favorable moment.
		\item[5] When you start a fire, be to windward of it. Do not attack from the leeward.
	\end{enumerate}
	\item A wind that rises in the daytime lasts long, but a night breeze soon falls.
	\item In every army, the five developments connected with fire must be known, the movements of the stars calculated, and a watch kept for the proper days.
	\item Hence those who use fire as an aid to the attack show intelligence; those who use water as an aid to the attack gain an accession of strength.
	\item By means of water, an enemy may be intercepted, but not robbed of all his belongings.
	\item Unhappy is the fate of one who tries to win his battles and succeed in his attacks without cultivating the spirit of enterprise; for the result is waste of time and general stagnation.
	\item Hence the saying: The enlightened ruler lays his plans well ahead; the good general cultivates his resources.
	\item Move not unless you see an advantage; use not your troops unless there is something to be gained; fight not unless the position is critical.
	\item No ruler should put troops into the field merely to gratify his own spleen; no general should fight a battle simply out of pique.
	\item If it is to your advantage, make a forward move; if not, stay where you are.
	\item Anger may in time change to gladness; vexation may be succeeded by content.
	\item But a kingdom that has once been destroyed can never come again into being; nor can the dead ever be brought back to life.
	\item Hence the enlightened ruler is heedful, and the good general full of caution. This is the way to keep a country at peace and an army intact.
\end{enumerate}

\chapter{The Use of Spies}
\begin{enumerate}
	\item Sun Tzu said: Raising a host of a hundred thousand men and marching them great distances entails heavy loss on the people and a drain on the resources of the State. The daily expenditure will amount to a thousand ounces of silver. There will be commotion at home and abroad, and men will drop down exhausted on the highways. As many as seven hundred thousand families will be impeded in their labor.
	\item Hostile armies may face each other for years, striving for the victory which is decided in a single day. This being so, to remain in ignorance of the enemy's condition simply because one grudges the outlay of a hundred ounces of silver in honors and emoluments, is the height of inhumanity.
	\item One who acts thus is no leader of men, no present help to his sovereign, no master of victory.
	\item Thus, what enables the wise sovereign and the good general to strike and conquer, and achieve things beyond the reach of ordinary men, is foreknowledge.
	\item Now this foreknowledge cannot be elicited from spirits; it cannot be obtained inductively from experience, nor by any deductive calculation.
	\item Knowledge of the enemy's dispositions can only be obtained from other men.
	\item Hence the use of spies, of whom there are five classes:
	\begin{enumerate}
		\item Local spies
		\item Inward spies
		\item Converted spies
		\item Doomed spies
		\item Surviving spies
	\end{enumerate}
	\item When these five kinds of spy are all at work, none can discover the secret system. This is called "divine manipulation of the threads." It is the sovereign's most precious faculty.
	\item Having local spies means employing the services of the inhabitants of a district.
	\item Having inward spies, making use of officials of the enemy.
	\item Having converted spies, getting hold of the enemy's spies and using them for our own purposes.
	\item Having doomed spies, doing certain things openly for purposes of deception, and allowing our spies to know of them and report them to the enemy.
	\item Surviving spies, finally, are those who bring back news from the enemy's camp.
	\item Hence it is that which none in the whole army are more intimate relations to be maintained than with spies. None should be more liberally rewarded. In no other business should greater secrecy be preserved.
	\item Spies cannot be usefully employed without a certain intuitive sagacity.
	\item They cannot be properly managed without benevolence and straightforwardness.
	\item Without subtle ingenuity of mind, one cannot make certain of the truth of their reports.
	\item Be subtle! be subtle! and use your spies for every kind of business.
	\item If a secret piece of news is divulged by a spy before the time is ripe, he must be put to death together with the man to whom the secret was told.
	\item Whether the object be to crush an army, to storm a city, or to assassinate an individual, it is always necessary to begin by finding out the names of the attendants, the aides-de-camp, and door-keepers and sentries of the general in command. Our spies must be commissioned to ascertain these.
	\item The enemy's spies who have come to spy on us must be sought out, tempted with bribes, led away and comfortably housed. Thus they will become converted spies and available for our service.
	\item It is through the information brought by the converted spy that we are able to acquire and employ local and inward spies.
	\item It is owing to his information, again, that we can cause the doomed spy to carry false tidings to the enemy.
	\item The end and aim of spying in all its five varieties is knowledge of the enemy; and this knowledge can only be derived, in the first instance, from the converted spy. Hence it is essential that the converted spy be treated with the utmost liberality.
	\item Hence it is only the enlightened ruler and the wise general who will use the highest intelligence of the army for purposes of spying and thereby they achieve great results. Spies are a most important element in water, because on them depends an army's ability to move.
\end{enumerate}

\end{document}